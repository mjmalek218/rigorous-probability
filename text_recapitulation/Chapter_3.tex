\documentclass{article}

\usepackage{amsmath}
\usepackage{amsfonts}
\usepackage[margin=0.5in]{geometry}



\begin{document}

\title{Chapter Three: Further Probabilistic Foundations}

\maketitle

\medskip 


\begin{abstract}

\noindent This chapter covers a slew of other relevant topics that form the basis of the theory, including random variables (which are essentially measurable functions), independence of random variables, and cases where different objects in the theory tail off to an infinite sequence. 

\end{abstract}

\bigskip

\section{Random Variables}

As part of the extension of the theory, we would like to examine functions which react based upon probabilistic outcomes: that is to say, a function which takes a value once the uncertainty of an event is settled. We call these ``random variables.''

\medskip

\noindent \textbf{Definition.} A \emph{random variable} is a function $X: \Omega \to R$ such that \[\{ \omega \in \Omega \mid X(\omega) \leq x \} \in \mathcal{F}, \; \forall x \in \mathbb{R}.\]

\medskip

This requirement is technical and necessary for the treatment of probability distributions later on. Essentially, for cumulative distributions functions we want there to be some way of measure the probabiliy that $X$ takes on a values less than or equal to a certain value $x \in \mathbb{R}$. In order to do so the image under the inverse mapping of these values needs to be a measurable set in $\mathcal{F}$: otherwise the probabiliy measure might not be defined. 

Now there are other ways of rephrasing this requirement. Recall that set operations are preserved under inverse mappings (due to the fact that they cannot map multiple elements in the range to the domain). Since the sigma algebra generated by the sets $(-\infty, x]$ is equivalent to the Borel sigma algebra generated by the open sets, we have that the above condition is akin to saying that for every Borel set $B$ in the reals, $X^{-1}(B) \in \mathcal{F}$. (But are \emph{countably} many set operations preserved by inverse mappings? I guess so... otherwise the book would be incorrect).

\medskip

\noindent \textbf{Proposition 3.1.5.} 

\begin{enumerate}[i] 

\item If $X = 1_A$ is the indicator of some event $A \in \mathcal{F}$, then $X$ is a random variable.

\item If $X$ and $Y$ are random variables and $c \in \mathbb{R}$ then $X + c$, $cX$, $X^2$, $X+Y$, and $XY$ are all random variables.

\item If $Z_1(\omega)$, $Z_2(\omega),...$ are random variables such that $\lim_{n \to \infty} Z_n(\omega)  = Z(\omega)$ for each $\omega \in \Omega$, then $Z$ is also a random variable. 

\end{enumerate}  


\medskip

\noindent \textbf{Proof.} \begin{enumerate}

\item 

\item 

\item 

\end{enumerate}



\end{document}