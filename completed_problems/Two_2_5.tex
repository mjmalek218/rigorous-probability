\documentclass{article}

\usepackage{amsmath}
\usepackage{amsfonts}

\usepackage[margin=0.5in]{geometry}

\begin{document}

\noindent \textbf{2.2.5} \textbf{Proposition.} Define \[\mathcal{B}_0 = \{\text{all finite unions of elements of J}\}\] where $J$ is defined as in the previous problem, being the collection of all intervals over $[0,1]$. Then $\mathcal{B}_0$ is an algebra but not a sigma algebra.

\medskip

\noindent \textbf{Proof.} Obviously $\emptyset$ and $\Omega$ are members. Now consider any two elements $A, B \in J$. Their intersection will be an intersection of intervals...since everything is finite it checks out. Also obviously closed under complements

The interest problem comes in proving that it is NOT a sigma algebra. For this we need to dig into our bag of tricks, so to speak. To prove this, we must demonstrate there exists a countable union of elements of $\mathcal{B}_0$ that is not a finite union of elements of $\mathcal{J}$: in other words, that it is not closed under the $\sigma$-algebra operations. Consider the set $\{ \frac{1}{n} \mid n \in \mathbb{N} \}$. This set must be in the corresponding sigma algebra but it is clearly not a finite union of intervals of the $[0,1]$. 

\hfill $\Box$

\end{document}