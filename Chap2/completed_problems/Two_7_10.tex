\documentclass{article}

\usepackage{amsmath}
\usepackage{amsfonts}
\usepackage[margin = .5in]{geometry}

\begin{document}

\noindent \textbf{Exercise Number: 2.7.10}  %% FILL THIS IN

\medskip 

\noindent \textbf{Proposition.} Letting $\Omega = \mathbb{R}$, define

\[\mathcal{J} = \Big\{(-\infty, x] \mid x \in \mathbb{R} \Big\} \cup \Big\{(y, \infty) \mid y \in \mathbb{R} \Big\} \cup \Big\{ (x,y] \mid x,y \in \mathbb{R} \Big\} \cup \Big\{\emptyset, \mathbb{R} \Big\}\]

where the union is taken over ``collections'' of sets (not a union in the strict set-theoretic
sense, I believe, though I need to learn more rigorous set theory to determine further). 

Then this may be a 

\bigskip

\noindent \textbf{Proof.} Recall the definition of a semi-algebra. $\mathcal{J}$ is a 
semi-algebra over an event space $\Omega$ if: 

\begin{enumerate}

\item $\Omega$, $\emptyset \in \mathcal{J}$

\item $A,B \in \mathcal{J} \Rightarrow A \cap B \in \mathcal{J}$ (i.e. $\mathcal{J}$ is closed under finite intersection 

\item For any $A \in \mathcal{J}$, $A^C$ is a union of disjoint sets in $\mathcal{J}$ 

\end{enumerate}

As a minor digression, note what this implies about unions in $\mathcal{J}$: consider $A,B \in \mathcal{J}$. Then by DeMorgan's Laws: 

\[A \cap B = (A^C \cap B^C)^C.\] Since $A^C, B^C$ are disjoint unions of sets in $\mathcal{J}$, 
and $\mathcal{J}$ is closed under intersection, we have that $A^C \cap B^C$ is a disjoint union
of sets of $\mathcal{J}$. Then taking the complement of this entire hodgepodge of sets must
again yield a disjoint union in the gaps of the sets (although further investigation may be necessary...).

\smallskip

\noindent Anyways, we verify the necessary conditions in the order presented:

\begin{enumerate}

\item Follows directly by the definition. 

\item Order the sets in the order presented in the union. If the sets in question
are disjoint, then they trivially yield a set of type 4. So assume not. Taking any intersection of
types 1 and 2 yields a set of type 3. Taking any intersection of type 1 and 3 yields a set of type 
3 again. Taking the intersection of sets 2 and 3 yields again a set of type 3. Finally if we
take the intersection of any set of type 4 with any other set, we either again get the empty set
or the other set stays the same (if $\mathbb{R}$ is used).   

\item Complement of set of type 1 is type 2, and vice versa. Complement of a set of type 3 is 
just a disjoint union of a set of type 1 and a set of type 2. Collections 4 are closed under
complements.  

\end{enumerate}

\hfill $\Box$


\bigskip

\noindent \textbf{Discussion.} 

Not much here, fairly intuitive. In fact this very semi-algebra, the set of intervals in $\mathbb{R}$, was the intuition originally provided to motivate the
creation of a $\sigma$-algebra via the extension theorem. Just a lot of book-keeping required. 

Note exactly what the collection $\mathcal{J}$ is defined to be here exactly: it is an amalgamation
of infinite intervals bounded from above by a hard number ($x$, note this type of interval
is key for the definition of cumulative distribution and the development of more applied theory
later on), the complement of these types of intervals, *half open intervals* that are open below
and closed above, and then $\Omega$ and the empty set. This is basically the simplest type
of semi-algebra that can contain the sets useful for cumulative distribution computation. Not by 
concidence, this semi-algebra is therefore used to derive the equivalence of measures on 
*all* borel sets when they are equivalent on all intervals of the type 1 (i.e. when their
cumulative distributions for any random variable defined as a continous function $\mathbb{R} \to
\mathbb{R}$ are the same, their distributions must be the same). 

\end{document}