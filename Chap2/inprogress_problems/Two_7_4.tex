\documentclass{article}

\usepackage{amsmath}
\usepackage{amsfonts}
\usepackage[margin = .5in]{geometry}

\begin{document}

\noindent \textbf{Exercise Number: 2.7.4}  %% FILL THIS IN

\medskip 

\noindent \textbf{Proposition.} Let $\mathcal{F}_1$, $\mathcal{F}_2$, ... be a sequence of collections of subsets of $\Omega$, such that $\mathcal{F}_n \subseteq \mathcal{F}_{n+1}$. 

\begin{enumerate}

\item Suppose each $F_i$ is an algebra. Then $\cup_{i=1}^\infty F_i$ is also an algebra.  

\item Now suppose each $F_i$ is a $\sigma$-algebra. Then $\cup_{i=1}^\infty F_i$ is not necessarily a $\sigma$-algebra. 

\end{enumerate}

\bigskip

\noindent \textbf{Proof.} 

\medskip

\noindent Recall the criteria required for a $\sigma$-algebra: finite additivity of sets, 
closed under complements, and includes the empty set. 

\begin{enumerate}

\item First note that we have the following fact: \[\cup_{i = 1}^n \mathcal{F}_i = \mathcal{F}_n \] since each successive algebra is just a refinement of the previous one. 

\item 

\end{enumerate} 

\hfill $\Box$

\end{document}