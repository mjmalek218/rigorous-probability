\documentclass{article}

\usepackage{amsmath}
\usepackage{amsfonts}
\usepackage[margin = .5in]{geometry}

\begin{document}

\noindent \textbf{Exercise Number: 2.7.9}  %% FILL THIS IN

\medskip 

\noindent \textbf{Proposition.} Let $\mathcal{F}$ be a $\sigma$-algebra, and write $|\mathcal{F}|$ to indicate the total number of sets in this collection. If $|\mathcal{F}|$ is finite, then $|\mathcal{F}| = 2^m$ for some $m \in \mathbb{N}$. 

\bigskip

\noindent \textbf{Proof.} We demonstrate a process by which all sets can be created from set operations of a certain ``basis'' so to speak, or basic collection, of sets in the $\sigma$-algebra. 

\bigskip

\noindent \textbf{Discussion.} This result at first seems striking, but with greater examination actually makes quite a bit of sense. The power-set collection of any set, for example, always has cardinality of 


\end{document}