\documentclass{article}

\usepackage{amsmath}
\usepackage{amsfonts}
\usepackage[margin = .5in]{geometry}

\begin{document}

\noindent \textbf{Exercise Number: 2.4.3}  %% FILL THIS IN

\bigskip 

\noindent \textbf{Proposition.} Define \[\mathcal{J} = \{\text{all intervals contained in } [0,1]\}.\] Let $\mathbb{P}$ be defined as the ``length'' of $I$. Then $\mathbb{P}$ as defined here is countably monotonic: that is to say, for any set $A \in \mathcal{J}$ and any countable collection of sets $\{A_i\}$ that are all subsets of $\mathcal{J}$ and $A \subseteq \cup_{\forall i} A_i$, we have \[P(A) \leq \sum_{\forall i} \mathbb{P}(A_i).\]

\bigskip

\noindent \textbf{Proof.} We demonstrate this in several steps, as outlined by the text. Let $I$ have endpoints $a < b$. 

\begin{enumerate}

\item First we demonstrate that if $I_1, I_2, ..., I_n$ is a finite collection of intervals, and if
$\cup _{j = 1}^n I_j \supseteq I$ for some interval $I$, then $\sum_{j = 1}^n \mathbb{P}(I_j) \geq \mathbb{P}(I)$. Since $\cup _{j = 1}^n I_j \supseteq I$, there must exist a subset of intervals $I_{\alpha_j}$ with $a_{\alpha_j} < b_{\alpha_j}$ being the endpoints, indexed by an index set $\alpha_j \in S$, cardinality $k$,  such that \[a_{\alpha_1} \leq a \leq  a_{\alpha_2} \leq b_{\alpha_1} < b_{\alpha_2} \cdots < b \leq b_{\alpha_k}.\] In other words, this subset of intervals encapsulates $I$. Then evidently, \[\mathbb{P}(I) \leq \sum_{j = 1}^k \mathbb{P}(I_{\alpha_j}) \leq \sum_{j = 1}^n \mathbb{P}(I_j).\]

\item Applying the Heine-Borel theorem to (1.) yields the result immediately. Heine-Borel is proved below. 

\item Define new intervals 

\end{enumerate}


\bigskip

\bigskip

\noindent \textbf{Lemma.} Heine-Borel theorem. If a colleciton of open-intervals contains a closed interval, then a finite sub-collection of these open intervals must contain the open interval as well. 

\medskip

\noindent \textbf{Proof.} 


\end{document} 