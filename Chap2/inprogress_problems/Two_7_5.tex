\documentclass{article}

\usepackage{amsmath}
\usepackage{amsfonts}
\usepackage[margin = .5in]{geometry}

\begin{document}

\noindent \textbf{Exercise Number: 2.7.5}  %% FILL THIS IN

\medskip 

\noindent \textbf{Proposition.} Suppose that $\Omega = \mathbb{N}$ is the set of positive integers 
and $\mathcal{F}$ is thet of all subsets of $A$ such that either $A$ or $A^c$ is finite, and 
$\mathbb{P}(A) = 0$ if $A$ is finite, and $\mathbb{P}(A) = 1$ is $A^C$ is finite. Then we have
the following:

\begin{enumerate}

\item $\mathcal{F}$ is an algebra. 

\item $\mathcal{F}$ is *not* a $\sigma$-algebra.

\item $\mathbb{P}$ is finitely additive. 

\item $\mathbb{P}$ is *not* countably additive. 

\end{enumerate}

\bigskip

\noindent \textbf{Proof.}  

\medskip

\begin{enumerate}

\item Closure under complements is trivial. Consider any two sets $A, B \in \mathcal{F}$. If we prove $A \cap B \in \mathbb{F}$ then by induction and DeMorgan's laws the result follows. The only way $A \cap C \notin \mathbb{F}$ would be if both $A \cap B$ *and* $(A \cap B)^C = A^C \cup B^C$ are both infinite (obviously they cannot both be finite given $\Omega$). Assume $A \cap B$ is infinite. Then at least one of $A$ or $B$ must be as well. If both are infinite, then $A^C \cup B^C$ is clearly finite and the result holds. Now assume WLOG that $A$ is infinite, but $B$ is finite. Then $A \cap B$ is clearly finite, and so consider $(A^C \cup B^C)$. Since $B^C$ is infinite this is infinite. Done. 

\item Define $A_n = \{2n\}$. Then clearly each $A_n \in \mathbb{F}$. However both $\cup_{n = 1}^{\infty} A_n$ *and* $\Big(\cup_{n = 1}^{\infty} A_n\Big)^C$ are clearly infinite. 

\item 

\end{enumerate}


\end{document}