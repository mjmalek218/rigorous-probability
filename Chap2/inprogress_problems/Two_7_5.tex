\documentclass{article}

\usepackage{amsmath}
\usepackage{amsfonts}
\usepackage[margin = .5in]{geometry}

\begin{document}

\noindent \textbf{Exercise Number: 2.7.5}  %% FILL THIS IN

\medskip 

\noindent \textbf{Proposition.} Suppose that $\Omega = \mathbb{N}$ is the set of positive integers 
and $\mathcal{F}$ is thet of all subsets of $A$ such that either $A$ or $A^c$ is finite, and 
$\mathbb{P}(A) = 0$ if $A$ is finite, and $\mathbb{P}(A) = 1$ is $A^C$ is finite. Then we have
the following:

\begin{enumerate}

\item $\mathcal{F}$ is an algebra. 

\item $\mathcal{F}$ is a $\sigma$-algebra.

\item $\mathbb{P}$ is finitely additive. 

\item $$

\end{enumerate}

\bigskip

\noindent \textbf{Proof.}  

\medskip

\begin{enumerate}

\item Closure under complements is trivial. Consider any two sets $A, B \in \mathcal{F}$. Note that it is impossible for $A$ and $B$ to both be disjoint and inifnite if they are in $\mathcal{F}$. TO see why, 

\item 

\item 

\end{enumerate}


\end{document}