\documentclass{article}

\usepackage{amsmath}
\usepackage{amsfonts}
\usepackage[margin = .5in]{geometry}
\usepackage{enumerate}

\begin{document}

\noindent \textbf{Exercise Number: 3.6.6}  %% FILL THIS IN

\bigskip

\noindent \textbf{Proposition.} Let $X$, $Y$, and $Z$ be three independent random variables, and set $W = X + Y$. Let \[B_{k,n} = \{(n-1) 2^{-k} \leq X < n 2^{-k}\}\] \[C_{k,m} = \{(m-1) 2^{-k} \leq Y < m 2^{-k}\}\] and let \[A_k = \bigcup_{\substack{n,m \in \mathbb{Z} \\ (n+m)2^{-k} < x}} (B_{k,n} \cap C_{k,m}).\] Finally, let $A = \{X + Y < x\}$ and $D = \{Z < z\}$. Then the following hold:

\begin{enumerate}[(a)]

\item $\{A_k\} \nearrow A$

\item $A_k$ and $D$ are independent 

\item $A$ and $D$ are independent 

\item $W$ and $Z$ are independent.

\end{enumerate}

\medskip

\noindent \textbf{Proof.} 

\begin{enumerate}[(a)]

\item First we demonstrate that $A_k$ is increasing. Take any arbitrary $k \in \mathbb{Z}$. Then we must show $A_k \subseteq A_{k + 1}$. Consider any $\omega \in A_k$. Then for some $n,m \in \mathbb{Z}$, $\omega \in B_{k,n} \cap C_{k,m}$, where $(n + m)2^{-k} < x$., and so \[\{(n-1) 2^{-k} \leq X(\omega) < n 2^{-k}\} \;\;\; \text{and} \;\;\; \{(m-1) 2^{-k} \leq Y(\omega) < m 2^{-k}\}.\] Writing these in terms of $2^{-(k + 1)}$ yields \[\{2(n-1) 2^{-k-1} \leq X(\omega) < (2n) 2^{-k-1}\} \;\;\; \text{and} \;\;\; \{2(m-1) 2^{-k-1} \leq Y(\omega) < (2m) 2^{-k-1}\}\] which immediately implies $\omega$ lies in some intersection of $B_{k+1,2n-1}$ or $B_{k+1,2n}$, and then $C_{k+1,2m-1}$ or $C_{k+1,2m}$ (choosing one of the $B$'s and one of the $C$'s). We need only demonstrate now that $(2n + 2m) 2^{-k - 1} < x$ (the worst case scenario), which follows immediately since $(n + m) 2^{-k}$ is true by assumption. 

Secondly we demonstrate that $\cup A_k = A$. The fact that $\cup A_k \subseteq A$ follows immediately, so we focus on the reverse. Consider any $\gamma \in \{\omega \mid (X+Y)(\omega) < x\}$. Let $a = X(\gamma)$ and $b = Y(\gamma)$ and  $a + b = c < x$. Since $2^{-k} \to 0$ as $k \to \infty$, there must exist a $p$ such that $2^{-p} < x - c$, which implies there exists a $z \in \mathbb{Z}$ such that $c < z 2^{-p} < x$. Thus \[A_z = \bigcup_{\substack{n,m \in \mathbb{Z} \\ (n+m)2^{-z} < x}} (B_{z,n} \cap C_{z,m})\] and, since the $B_{z,n}$ and $C_{z,m}$ each separately partition the set $[0, 2^{z})$, there is some $n$ and $m$ satisfying these conditions such that $(n-1)2^{-z} \leq a < n 2^{-z}$ and $(m-1)2^{-z} \leq b < m 2^{-z}$, and so we finally have $\gamma \in A_z$. 


\item Consider $\mathbb{P}(A_k \cap D)$. By definition this equals  \[\mathbb{P}\Bigg[ D \cap \Big( \bigcup_{\substack{n,m \in \mathbb{Z} \\ (n+m)2^{-k} < x}} (B_{k,n} \cap C_{k,m}) \Big) \Bigg].\] By DeMorgan's Laws and/or basic set theory, we may rewrite as \[\mathbb{P}\Bigg[ \bigcup_{\substack{n,m \in \mathbb{Z} \\ (n+m)2^{-k} < x}} (D \cap B_{k,n} \cap C_{k,m}) \Bigg].\] Now note that each of the $(D \cap B_{k,n} \cap C_{k,m})$ are disjoint from another, because they each are subsets of respective partitions of $\Omega$ (the partitions being the $B$'s and the $C$'s, partitioned by how $X$ and $Y$ respectively map to the reals) and so we may apply countable subadditivity of your measure to yield  \[\mathbb{P}\Bigg[ \bigcup_{\substack{n,m \in \mathbb{Z} \\ (n+m)2^{-k} < x}} (D \cap B_{k,n} \cap C_{k,m}) \Bigg] = \sum_{\substack{n,m \in \mathbb{Z} \\ (n+m)2^{-k} < x}} \mathbb{P}(D \cap B_{k,n} \cap C_{k,m})\] which, since $D$, $B_{k,n}$, $C_{k,m}$ are all Borel and therefore independent (by virtue of being defined by $X$, $Y$, and $Z$, respectively and exclusively), is equivalent to \[\sum_{\substack{n,m \in \mathbb{Z} \\ (n+m)2^{-k} < x}} \mathbb{P}(D) \mathbb{P}(B_{k,n} \cap C_{k,m}) = \mathbb{P}(D) \mathbb{P} \Big[\bigcup_{\substack{n,m \in \mathbb{Z} \\ (n+m)2^{-k} < x}} B_{k,n} \cap C_{k,m} \Big] = \mathbb{P}(D) \mathbb{P}(A_k).\]

\item By $(a)$, $(b)$, and continuity of probabilities (monotonically increasing/decreasing sequences of events), \[\mathbb{P}(A \cap D) = \mathbb{P}(\lim_{k \to \infty} A_k \cap D) = \lim_{k \to \infty} \mathbb{P}(A_k) \mathbb{D} = \mathbb{P}(A) \mathbb{P}(D). \]

\item By definition, $W$ and $Z$ are independent if for all borel sets $S_1, S_2$, $W^{-1}(S_1)$ and $Z^{-1}(S_2)$ are independent events. By Chap 2 we know the sets $(-\infty, x]$ generate all Borel sets, so setting $A_x = \{X + Y < x\}$ and $D_z = \{Z < z\}$ and remembering that inverse images preserve all set operations under which $\sigma$-algebras are closed, there must exist countable collections of $A_x$ and $D_z$ generating $W^{-1}(S_1)$ and $Z^{-1}(S_2)$, respectively. By Lemma 3.5.2 these events are still independent since they are composed on independent events, completing the proof. 

\end{enumerate}

\hfill $\Box$

\bigskip

\noindent \textbf{Discussion.} 

\smallskip

The overall implication of this problem is that if we sum independent random variables, the sum of these random variables retain the exogenous independence properties of their components. The general method of proving this was complex, but after some thought follows as natural from the definition. Looking at it top down...(with the ``scaffolding'' so to speak...)

We wanted to prove that $X + Y$ and $Z$ are independent. In general, random variables are independent if and only if on Borel sets the inverse images are independent. We can construct Borel sets from unbounded lower intervals of numbers (in the spirit of distributions and the original definition of random variables) and, since inverse images of sets preserve all relevant set operations, we work backwards to prove that the inverse functions of $W$ and $Z$ map to .

Then we can break our problem down further: if we demonstrate on an increasing/decreasing sequence of finite sets that the inverse image on these sets is again independent of $D$ (as defined above) then we can apply continuity of probabilities to achieve our final result. In our case it just so happens an increasing sequence was chosen, and is perhaps more natural. 

So essentially, whoever came up with this proof template must have worked backwards in the following manner. The biggest leap in the reverse engineering (top-down approach), in my humble opinion, must have been the the construction of the $B$ and $C$ sets to break down $A_k$ further. But looking at the solution, it seems fairly natural tbh. 

\end{document} 