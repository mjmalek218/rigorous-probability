\documentclass{article}

\usepackage{amsmath}
\usepackage{amsfonts}
\usepackage[margin = .5in]{geometry}

\begin{document}

\noindent \textbf{Exercise Number: 3.6.1}  %% FILL THIS IN

\begin{enumerate}

\item $\mathcal{F}$ is a collection of subsets of $\Omega$.

\item $\mathbb{P}(A)$ is a well-defined element of $\mathbb{R}$ provided that $A$ is an element of $\mathcal{F}$. 

\item $\{X \leq 5\}$ is shorthand notation for the particular subset of $\Omega$ which is defined by: $\{\omega \in \Omega \mid X(\omega) \leq 5\}$.

\item If $S$ is a subset of $\mathbb{R}$, then $\{X \in S\}$ is a subset of $\Omega$.

\item If $S$ is a Borel subset of $\mathbb{R}$, then $\{X \in S\}$ must be an element of $\mathcal{F}$. 

\end{enumerate}



\end{document} 