\documentclass{article}

\usepackage{amsmath}
\usepackage{amsfonts}
\usepackage[margin = .5in]{geometry}

\begin{document}

\noindent \textbf{Exercise Number: 3.2.2}  %% FILL THIS IN

\bigskip

\noindent \textbf{Proposition.} Consider a possibly infinite collection $\{A_{\alpha}\}_{\alpha \in I}$
of events. Suppose they are independent: i.e. for each $j \in \mathbb{N}$ and each distinct finite choice $\alpha_1, ..., \alpha_j$ the following holds: \[\mathbb{P}(A_{\alpha_1} \cap A_{\alpha_2} \cap \cdots \cap A_{\alpha_j}) = \mathbb{P} (A_{\alpha_1}) \mathbb{P} (A_{\alpha_2}) \cdots \mathbb{P}( A_{\alpha_j}).\] Then if any arbitrary $A_{\alpha_i}$ is replaced by $A_{\alpha_i}^C$, the independence property still holds. Logically this implies any arbitrary number of the events may be replaced by their complements, and independence will still hold. 

\bigskip

\noindent \textbf{Proof.} WLOG let $i = 1$ in the proposition, and let $B =  A_{\alpha_2} \cap \cdots \cap A_{\alpha_j}$. Then by countable additivity of our probability measure the following must hold: \[\mathbb{P}(A_{\alpha_1} \cap B) + \mathbb{P}(A_{\alpha_1}^C \cap B)  =\mathbb{P}(B).\] By the proposition statement this then implies \[\mathbb{P}(A_{\alpha_1}^C \cap B)  = \mathbb{P}(B) - \mathbb{P}(A_{\alpha_1} \cap B) =  \mathbb{P} (A_{\alpha_2}) \cdots \mathbb{P}( A_{\alpha_j}) - \mathbb{P} (A_{\alpha_1}) \mathbb{P} (A_{\alpha_2}) \cdots \mathbb{P}( A_{\alpha_j}) = \] \[\mathbb{P} (A_{\alpha_2}) \cdots \mathbb{P}( A_{\alpha_j}) * (1 - \mathbb{P}(A_{\alpha_1})) = \mathbb{P} (A_{\alpha_1}^C) \mathbb{P} (A_{\alpha_2}) \cdots \mathbb{P}( A_{\alpha_j})\] as desired. 

\hfill $\Box$


\end{document}