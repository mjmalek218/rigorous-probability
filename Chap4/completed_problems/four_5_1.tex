\documentclass{article}

\usepackage{amsmath}
\usepackage{amsfonts}
\usepackage[margin = .5in]{geometry}

\begin{document}

\noindent \textbf{Exercise Number: 4.5.1}  %% FILL THIS IN

\medskip 

\noindent \textbf{Proposition.} Let $(\Omega, \mathcal{F}, \mathbb{P})$ be the Lebesgue measure on $[0,1]$ and set \[X(\omega) = \begin{cases}
                                    1, \;\; 0 \leq \omega \leq 1/4 \\
                                    2\omega^2, \;\; \frac{1}{4} \leq \omega \leq 3/4 \\
                                    \omega^2, \;\; \frac{3}{4} \leq \omega \leq 1 \\
                                    \end{cases}.\] Then 

\begin{enumerate}

\item $\mathbb{P}(X \in A) = \frac{1}{4} + \sqrt{\frac{1}{2}}$ if $A = [0,1]$. 

\item $\mathbb{P}(X \in A) = \frac{3}{4}$  if $A = [\frac{1}{2}, 1]$.

\end{enumerate}

\bigskip

\noindent \textbf{Proof.} This is just a matter of bookkeeping. Since the Lebesgue measure of
intervals is just their length, our job is simple in this case. We calculate the cases in the
order they were presented.  

\begin{enumerate}

\item $\mathbb{P}(X \in A) = X^{-1}([0,1]) = 1 - X^{-1}(\mathbb{R} \setminus [0,1])$. Looking at 
the definition of $X$ quickly revels that the only portion that maps to outside 1, maps above it:
for $2\omega^2$. Since it is monotonically increasing just set \[2\omega^2 = 1 \;\; \Rightarrow \;\; \omega = \sqrt{\frac{1}{2}}.\] Thus \[\mathbb{P}(X \in A) = 1 - (\frac{3}{4} - \sqrt{\frac{1}{2}}) = \frac{1}{4} + \sqrt{\frac{1}{2}}.\] 

\item In the second case we choose to calculate directly rather than via complement for convenience. \[\mathbb{P}(X \in A) = X^{-1}([\frac{1}{2},1]) = (\frac{1}{4} - 0) + (\frac{3}{4} - \frac{1}{2}) + (1 - \frac{3}{4}) = 3 * \frac{1}{4} = \frac{3}{4}.\] 

\end{enumerate}  


\hfill $\Box$


\end{document} 