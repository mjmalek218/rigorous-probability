\documentclass{article}

\usepackage{amsmath}
\usepackage{amsfonts}
\usepackage[margin = .5in]{geometry}

\begin{document}

\noindent \textbf{Exercise Number: 4.3.2}  %% FILL THIS IN

\medskip 

\noindent \textbf{Proposition.} Let $X$ and $Y$ be two general random variables (not necessarily non-negative) with well-defined means, such that $X \leq Y$. Then 

\begin{enumerate}

\item $X^+ \leq Y^+$ and $X^- \geq Y^-$.

\item Expectation is still order-preserving: i.e. $\mathbb{E}(X) \leq \mathbb{E}(Y)$ under these assumptions. 

\end{enumerate}

\bigskip

\noindent \textbf{Proof.} The proposition implies that $\forall \omega \in \Omega$, $X(\omega) \leq Y(\omega)$. So obviously if we restrict to positive and negative ranges of these functions, the first result follows. 

The second portion of the proposition just follows from the order preserving property of non-negative random variables applied to the first portion: $\mathbb{E}(X^+) \leq \mathbb{E}(Y^+)$ and $\mathbb{E}(X^-) \geq \mathbb{E}(Y^-)$ and so by definition of general expectation the result holds. 

\hfill $\Box$



\end{document} 