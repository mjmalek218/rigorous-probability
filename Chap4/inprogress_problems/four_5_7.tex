\documentclass{article}


\usepackage{amsmath}
\usepackage{amsfonts}
\usepackage[margin = .5in]{geometry}


\begin{document}

\noindent \textbf{Exercise Number: 4.5.7}  %% FILL THIS IN

\medskip 

\noindent \textbf{Proposition.} \emph{(Principle of inclusion-exclusion, general case)} Let $A_1$, $A_2$, ..., $A_n \in \mathcal{F}$. Then the general principle of inclusion-exclusion holds, i.e.:

\[\mathbb{P}(A_1 \cup A_2 \cup \cdots \cup A_n) = \sum_{i=1}^n \mathbb{P}(A_i) - \sum_{1 \leq i < j \leq n} \mathbb{P}(A_i \cap A_j) + \sum_{1 \leq i < j < k \leq n} \mathbb{P}(A_i \cap A_j \cap A_k) - \cdots  \pm \mathbb{P}(A_1 \cap \cdots \cap A_n)\] 

\bigskip

\noindent \textbf{Proof.} To prove the general inclusion-exclusion formula efficiently, we exploit
a convenient trick. We write $P(A_1 \cup \cdots \cup A_n)$ in a convenient form (expectation combined with indicator variables), and then expand it into two different forms, applying the expectation
differently. A trivial result of expectation/indicator vairables is the following: $\mathbb{E}(1_A) = \mathbb{P}(A)$. Thus, we have \[\mathbb{P}(A_1 \cup \cdots \cup A_n) = \mathbb{E}(1_{A_1 \cup \cdots \cup A_n}) = \mathbb{E}(1 - 1_{(A_1 \cup \cdots \cup A_n)^c}) = \mathbb{E}(1 - 1_{A_1^c \cap \cdots \cap A_n^c})  =  \]


\end{document} 