\documentclass{article}


\usepackage{amsmath}
\usepackage{amsfonts}
\usepackage[margin = .5in]{geometry}
\usepackage{enumerate}

\begin{document}

\noindent \textbf{Exercise Number: 4.3.4}  %% FILL THIS IN

\medskip 

\noindent \textbf{Proposition.} Let $X$ and $Y$ be two general independent random variables with finite means, and let $Z = XY$. Then the following holds true:

\begin{enumerate}[(a)]

\item $X^{\#_x}$ and $Y^{\#_y}$ are independent, where $\#_x, \#_y \in \{+,-\}$

\item $Z^+ = X^+ Y^+ + X^- Y^-$ and  $Z^- = X^- Y^+ + X^+ Y^-$

\item $\mathbb{E}(Z) = \mathbb{E}(X)  \mathbb{E}(Y)$. 

\end{enumerate}

\bigskip

\noindent \textbf{Proof.} Proven in the order presented.  

\begin{enumerate}

\item Consider any two borel sets $B_x, B_y \subseteq \mathbb{R}$. We are given that \[\mathbb{P}[X^{-1}(B_x) \cap Y^{-1}(B_y)] = \mathbb{P}[X^{-1}(B_x)] \mathbb{P}[Y^{-1}(B_y)].\] Now consider \[X^{-1}(B_x) = (X^+ - X^-)^{-1}(B_x).\] The claim is this may be broken up into the union of 

\item Fairly straight forward: details fairly trivial here. $Z^+$ is going to on the omega where both $X$ and $Y$ are positive or negative: sinceit is defined as a product. When one omega maps both to negative, $X^+ Y^+$ is 0 and vice versa, allowing the desired result. Same logic applies to $Z^-$.
 
\item Just apply the results 

\end{enumerate}

\hfill $\Box$


\bigskip

\noindent \textbf{DISCUSSION:} The point of this proposition (and the previous one) is that
this new general definition of expectation which involves its decomposition into negative/positive
components really does not change the properties of expectation that had been previously examined
...in a ``finite'' setting. 

\end{document} 