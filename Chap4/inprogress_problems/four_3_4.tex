\documentclass{article}


\usepackage{amsmath}
\usepackage{amsfonts}
\usepackage[margin = .5in]{geometry}


\begin{document}

\noindent \textbf{Exercise Number: 4.5.3}  %% FILL THIS IN

\medskip 

\noindent \textbf{Proposition.} Let $X$ and $Y$ be two general random variables with finite means, and let $Z = X + Y$. 

\begin{enumerate}

\item $Z^+ - Z^- = X^+ - X^- + Y^+ - Y^-$. 

\item $\mathbb{E}(Z) = \mathbb{E}(X) + \mathbb{E}(Y)$

\item The general definition of expectation is finitely linear, for general random variables with finite means. 

\end{enumerate}

\bigskip

\noindent \textbf{Proof.} Proven in the order presented.  

\begin{enumerate}

\item Follows by definition of expectation. 

\item We may re-arrange the first relation to write \[Z^+ + X^- + Y^- = X^+  + Y^+ + Z^+.\] Since these are positive random variables with positive coefficients, we may apply the result of equation 4.2.6 of the chapter to write \[\mathbb{E}(Z^+) + \mathbb{E}(X^-) + \mathbb{E}(Y^-) = \mathbb{E}(X^+)  + \mathbb{E}(Y^+) + \mathbb{E}(Z^+)\] which through trivial re-arranging gives the desired result. 

\item From (2) it is clear as to why this general definition of expectation is finitely linear. Consider any arbitrary linear combination of random variables $X_i$ with scalars $a_i$: $Z = \sum_{i = 1}^n a_i X_i$. If any of the scalars are negative, we may simply switch them to the other side of the equation and apply the same logic as in (2): all that is needed is positive scalar multiples. 

\end{enumerate}

\hfill $\Box$

\end{document} 