\documentclass{article}

\usepackage{amsmath}
\usepackage{amsfonts}

\begin{document}

\textbf{1.3.1}

\medskip 

\noindent \textbf{Proposition.} Suppose $\Omega = {1,2}$ with $\mathbb{P}(\emptyset) = 0$ and $\mathbb{P}(\{1,2\}) = 1$. Suppose $\mathbb{P} = \frac{1}{4}$. Then $\mathbb{P}$ is countably additive if and only if $\mathbb{P}(\{2\}) = \frac{3}{4}$.  
\bigskip

\noindent \textbf{Proof.} First assume $\mathbb{P}$ is countably additive. Then \[\mathbb{P}(\{1,2\}) = \mathbb{P}(\{1\}) + \mathbb{P}(\{2\}) = 1/4 + \mathbb{P}(\{2\})  = 1 \Rightarrow \mathbb{P}(\{2\}) = 3/4.\]

Next assume the reverse. Then again since there are only two elements in the set the result follows trivially. 


\hfill $\Box$

\end{document}
