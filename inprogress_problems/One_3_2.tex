\documentclass{article}

\usepackage{amsmath}
\usepackage{amsfonts}

\begin{document}

\textbf{Exercise Number: 1.3.2}  %% FILL THIS IN

\medskip 

\noindent \textbf{Proposition.} Suppose $\Omega = \{1,2,3\}$ and $\mathcal{F}$ is the collection of all subsets of $\Omega$. The following are the necessary and sufficient conditions on the real numbers $x,y,$ and $z$ such that there exists a countably additive probability measure $\mathbb{P}$ on $\mathcal{F}$, with $x = \mathbb{P}\{1,2\}$, $y = \mathbb{P}\{2,3\}$, $z = \mathbb{P}\{1,3\}$:

\begin{enumerate}

\item $x+ y + z = 2$

\item Triangle inequality on $x,y,z$ must hold true. 

\item $x,y,z$ must all be non-negative. 

\end{enumerate}

\bigskip

\noindent \textbf{Proof.} We first prove that individually all these conditions are necessary, and then that they are sufficient. 

\noindent Necessary:

\begin{enumerate}

\item Since $\mathbb{P}$ is countably additive, we have that $\mathbb{P}\{1,2\} = \mathbb{P}\{1\} + \mathbb{P}\{2\} = x$, etc... for each $x,y,$ and $z$. Summing all of these out yields this fact simply since $\mathbb{P}(\Omega) = 1$.

\item This holds trivially. 

\item Obviously since $P$ is a probability measure all of these values must be greater than 0. 

\item This holds trivially since all are sums of  

\end{enumerate}


\noindent Sufficient: 

Now we assume the conditions to be true and derive that $\mathbb{P}$ is a countably additive probability measure from them, given the conditions on $x,y,$ and $z$. 






\end{document}